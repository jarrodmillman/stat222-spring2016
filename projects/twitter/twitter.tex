\documentclass[11pt, oneside]{article}   	% use "amsart" instead of "article" for AMSLaTeX format
\usepackage{geometry}                		% See geometry.pdf to learn the layout options. There are lots.
\geometry{letterpaper}                   		% ... or a4paper or a5paper or ... 
%\geometry{landscape}                		% Activate for rotated page geometry
%\usepackage[parfill]{parskip}    		% Activate to begin paragraphs with an empty line rather than an indent
\usepackage{graphicx}				% Use pdf, png, jpg, or eps§ with pdflatex; use eps in DVI mode
								% TeX will automatically convert eps --> pdf in pdflatex		
\usepackage{amssymb}

%-----------------------------------------------------------------------------
% Special-purpose color definitions (dark enough to print OK in black and white)
\usepackage{color}
% A few colors to replace the defaults for certain link types
\definecolor{orange}{cmyk}{0,0.4,0.8,0.2}
\definecolor{darkorange}{rgb}{.71,0.21,0.01}
\definecolor{darkgreen}{rgb}{.12,.54,.11}
%-----------------------------------------------------------------------------
% The hyperref package gives us a pdf with properly built
% internal navigation ('pdf bookmarks' for the table of contents,
% internal cross-reference links, web links for URLs, etc.)
\usepackage{hyperref}
\hypersetup{pdftex, % needed for pdflatex
  breaklinks=true, % so long urls are correctly broken across lines
  colorlinks=true,
  urlcolor=blue,
  linkcolor=darkorange,
  citecolor=darkgreen,
}

\usepackage{booktabs}


\title{Stat 222: Twitter Project}
\date{}							% Activate to display a given date or no date

\begin{document}
\maketitle

\section{Data Description}

For this project, your primary data source will be Twitter.  And you will be
expected to work with the Twitter API using Python to access this data. On
Monday (2/23), I will briefly review Python, introduce JSON (Javascript Object
Notation), and demonstrate how to interact with the Twitter API using the
\href{https://github.com/sixohsix/twitter}{Python Twitter Tools}.  However, you
will need to do outside reading to get up to speed with these tools. 

Python Twitter Tools is one of several Python
packages\footnote{\url{http://www.danielforsyth.me/analyzing-a-nhl-playoff-game-with-twitter}}
for interacting with the Twitter API.  It is fairly minimal and is the package
used in chapter 1 and 9 of \emph{Mining the Social
Web}.\footnote{\url{https://github.com/ptwobrussell/Mining-the-Social-Web-2nd-Edition}}
\begin{itemize}
\item \href{https://rawgit.com/ptwobrussell/Mining-the-Social-Web-2nd-Edition/master/ipynb/html/Chapter%201%20-%20Mining%20Twitter.html}{Chapter 1: Mining Twitter: Exploring Trending Topics, Discovering What People Are Talking About, and More}
\item \href{https://rawgit.com/ptwobrussell/Mining-the-Social-Web-2nd-Edition/master/ipynb/html/Chapter%209%20-%20Twitter%20Cookbook.html}{Chapter 9: Twitter Cookbook}
\item \href{http://nbviewer.ipython.org/github/ptwobrussell/Mining-the-Social-Web-2nd-Edition/tree/master/ipynb/}{Mining the Social Web notebooks}
\end{itemize}


\section{Your Assignment}

Each group is responsible for creating a poster presentation that visually
answers a set of questions, which you will determine for yourselves.  You will
first need to decide on a set of question that you can use Twitter data to
answer.  Once you determine the questions you wish to address, you will need to
use the Python Twitter package to download the data from Twitter that you will
use to answer the question.  You are free to use Python to analyze the data and
create plots.  Since you've had limited practice with Python, you are welcome
to use R for some of the analysis and for creating your figures. However, even
if you decide to use R for some of the analysis and plotting, you must use
Python to retrieve the data as well as most of the preprocessing.  For example,
you may decide to use Python to create a term-document matrix (or a term
frequency by inverse document frequency matrix) and save the data as a CSV
file, which you can then read with R.

In terms of number of plots in your poster, a {\em minimum} number to aim for
is 10, although it might be natural for some plots to be grouped together into
a single figure. Last year, Cari Kauffman and I created a slide presentation as
an example of the kind of questions and visual answers you might use.  You
should look over our example, although you should not simply replicate our
plots.  You can find our slides \href{here}{http://jarrodmillman.com} and
a transcript of our talk \href{here}{http://jarrodmillman.com}.

{\em Which} questions you answer is up to you, but think about telling a story.
The story will be more interesting if the questions you address are related to
each other in some way.  Here are a few example topics:

\begin{itemize}
\item investigate the relation of breaking news on Twitter versus traditional
      news sources
\item compare stop words usage on Twitter versus NY Times
\item chart how the ratio of positive versus negative words used in tweets
      involving some event (or issue) change over time
\item relate tweets about a TV show/movie/book to their viewers/ticket
      sales/sales over time
\item examine tweets containing a specific hashtag
\item compare tweets relating to competing atheletic teams
\item look at tweets from academics perhaps relating to a conference
      or journal article
\end{itemize}

Here are a few things to keep in mind as you think about what questions you
want to ask and how you might address them.

\begin{itemize}

\item You will be limited in the amount of historical data you can retrieve, so
questions about how things evolve over time may difficult to investigate.  I
will show you how to use cron jobs running on a server to run your queries
automatically on a regular schedule.  This will allow you to somewhat overcome
the data limitations imposed by Twitter policy.  For example, if Twitter only
provides responses for the last two weeks, then using a cron job you can
automatically retrieve two weeks of data every week (or two).

\item Since tweets are short texts, text mining is a natural approach to take.
However, it isn't the only one.  For instance, you might create a graph where
the nodes are twitter accounts and the edges represent which accounts follow
which accounts. Using this graph you could try to identify hubs, authorities,
and communities.

\item While the focus of your project will involve Twitter data, you should
consider incorporating external data sources.  For example, you may want to
relate your findings to current events (e.g., a spike in activity may be
associated with major news events occuring just before the spike).  Or,
perhaps, you will want to get additional data about certain Twitter users
(e.g., pulling biographical details about a Twitter user from Wikipedia or
their homepage) or organization (e.g., pulling 

\end{itemize}

\subsection*{Timeline and logistics}

Here is the tentative schedule:

\begin{table}[h]
\centering
\begin{tabular}{@{}l|l@{}}
\toprule
\multicolumn{1}{c|}{Monday} & \multicolumn{1}{c}{Wednesday} \\
\hline
(1/18) No class              & (1/20) Start Twitter project \\
(1/25) Text mining I         & (1/27) Text mining II \\
(2/1) Graphics I             & (2/3) Graphics II \\
(2/8) Slide presentations    & (2/10) Group work \\
(3/15) No class              & (3/17) Group work \\                                       
(3/22) Poster presentations  & (3/24) Start group project 2 \\                                       
\bottomrule
\end{tabular}
\end{table}

Over winter break you should spend some time exploring Twitter
(\url{http://www.twitter.com}) to familiarize yourself with how it works.
Explore the Twitter interface to discover who uses it and how. You may wish to
beginning following a few accounts.  Pay attention to how people use
\texttt{\#hashtags} and \texttt{\ \@mentions}.  In addition to individuals, look at
what organizations (e.g., companies, non-profits, political campaigns) use it
and how.  While doing this, you should also begin thinking about what kind of
questions you may want to address.

I recommend that you not focus on what methods you would use at this point.
Once the semester begins, I will discuss how to collect Twitter data as well as
several different analysis approaches you might take.

I will randomly assign groups during the first day of class. Prior to the
following Monday, your team should meet to discuss what general questions you
want to ask.  Once you have a general idea of what questions you want to ask,
you need to decide what data you want to collect.  You should immediately start
collecting Twitter data with the goal of addressing your questions.  (You will
not be permitted to use data that you collect prior to the start of the
semester.)  In addition to the Twitter data, your group should also discuss
what other data sources you may need to use.

Next, working individually over the weekend, brainstorm {\em three} specific
questions to use a starting point for exploring this data. The answers to these
questions will almost certainly suggest other questions to you, but use these
as a starting point.  For each question,
\begin{itemize}
\item Think about what plot you could make to help answer the question.
\item Describe exactly what data or data summaries you need to make the plot.
\item Make an initial version of the plot. (You may want to refine it later.)
\item Interpret the plot. Does it suggest any other questions/plots to consider?
\end{itemize}
These steps may sound obvious, but I know from experience that the temptation
is there to jump in and start writing code before you've given much thought to
what you want to learn. I think you'll find you get richer and more interesting
results in an open-ended project like this if you allocate more of your time at
this stage to {\em thinking} and less to implementation.

During the second week of the semester, I will discuss text mining in Python.
This will include basic summary statistics (e.g., word frequency, concordance,
collocation), the bag-of-words model, document similarity metrics, latent
semantic indexing, sentiment analysis, and topic modeling.  

using the Natural
Language Toolkit (nltk).\footnote{http://www.nltk.org} By this point, your
group should have downloaded the Twitter data and whatever other data you
think you will need.  

\section{Initial Guidance (to do for first data debrief)}

Since this is a project for which you'll have to define and obtain the data
yourselves, the goal for the first week is for you \emph{to define what tweets
(or other info) you want to work with, download that data, and wrangle it into
a simpler format}. 

An important goal for this project is to provide an opportunity for you to get
more practice using Python.  In particular, for this project I expect you to
gain more experience working with basic Python structures (lists, dictionaries,
tuples, and strings) and work with JSON and CSV using Python.  You will also be
using Python's string processing and text mining capabilities to process the
data.  While you may wish to use some of the packages in the scientific Python
stack, the focus will be on core Python language features and a few specialized
libraries.


\section{Presentation Details}

The main deliverable will be a poster.  You 
1. Slide deck

2. Poster draft to submit

3. Incorporate new data

4. Possibly present at BSTARS

\section{Poster Details}

There are many ways to create a poster. Here are some possibilities for you to explore:
\begin{itemize}
\item Beamer Poster Package\\
 \url{http://www-i6.informatik.rwth-aachen.de/~dreuw/latexbeamerposter.php}
\item Powerpoint (try searching for "research poster powerpoint template")
\item Adobe Illustrator or InDesign (available free to students at \url{https://software.berkeley.edu/adobe})
\item Pages (recommended over Keynote for posters)
\end{itemize}

Your posters will be due {\em at the latest} on Tuesday, February 24, and we'll
spend some time sharing them in class on Wednesday, Februrary 25. (You should
have everything finished by Monday, so that you have time for printing.) A
service I can recommend is
\url{http://gif.berkeley.edu/services/printing.html}. Note that they require an
appointment, which I recommend you go ahead and schedule now.

If you use this printing service, your poster should be 36" or 42" along one
side. (These are the paper sizes they stock.) 36" high x 48" wide is fairly
common poster size.

When designing your poster, you may find it helpful to draw the layout before
you try to implement it in software. Another piece of advice is to be careful
with the resolution of your figures. After designing your poster, you may need
to regenerate them at the actual size they will occupy. Taking a small figure
and enlarging it to fit the poster will cause the image quality to be poor.

The poster should contain a title, your names, data source(s), the plots, and
text to tie everything together. Someone should be able to understand the main
findings simply by reading it, but be careful not to include so much text that
the poster becomes difficult to skim. It's typical to ``present'' the poster and
give more details verbally. 

\end{document}  
