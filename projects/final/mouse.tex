\documentclass[11pt, oneside]{article}   	% use "amsart" instead of "article" for AMSLaTeX format
\usepackage{geometry}                		% See geometry.pdf to learn the layout options. There are lots.
\geometry{letterpaper}                   		% ... or a4paper or a5paper or ... 
%\geometry{landscape}                		% Activate for rotated page geometry
%\usepackage[parfill]{parskip}    		% Activate to begin paragraphs with an empty line rather than an indent
\usepackage{graphicx}				% Use pdf, png, jpg, or eps§ with pdflatex; use eps in DVI mode
								% TeX will automatically convert eps --> pdf in pdflatex		
\usepackage{amssymb}
\usepackage{upquote}

%-----------------------------------------------------------------------------
% Special-purpose color definitions (dark enough to print OK in black and white)
\usepackage{color}
% A few colors to replace the defaults for certain link types
\definecolor{orange}{cmyk}{0,0.4,0.8,0.2}
\definecolor{darkorange}{rgb}{.71,0.21,0.01}
\definecolor{darkgreen}{rgb}{.12,.54,.11}
%-----------------------------------------------------------------------------
% The hyperref package gives us a pdf with properly built
% internal navigation ('pdf bookmarks' for the table of contents,
% internal cross-reference links, web links for URLs, etc.)
\usepackage{hyperref}
\hypersetup{pdftex, % needed for pdflatex
  breaklinks=true, % so long urls are correctly broken across lines
  colorlinks=true,
  urlcolor=blue,
  linkcolor=darkorange,
  citecolor=darkgreen,
}

\usepackage{booktabs}


\title{Stat 222: Mouse Behavior Project}
\date{}							% Activate to display a given date or no date

\begin{document}
\maketitle

\section{Data Description}

For this project, your primary data source will be mouse behavioral data from
the Tecott Lab at
UCSF.\footnote{http://www.neuroscience.ucsf.edu/neurograd/faculty/tecott.html}
The lab has recently developed a method for continuous high-resolution
behavioral data collection and analysis, which enables them to observe and
study the structure of spontaneous patterns of behavior (``Lifestyles'') in the
mouse.  They have found that using this method: 1) reveals a set of fundamental
principles of behavioral organization that have not been previously reported,
2) permits classification by genotype with unprecedented accuracy, and 3)
enables fine dissection of behavioral patterns.

A central goal for this project is to give you hands-on experience working
with ...

\section{Your Assignment}

Working as a fictitious consulting company, Capstone Analytics, we will produce
a high quality Python package to analyze some of the new data generated in the
Tecott lab.

I (Jarrod) will act as the head of Capstone Analytics and you will all act as
employees. 

\subsection*{Timeline and logistics}

Here is the tentative schedule:

\begin{table}[h]
\centering
\begin{tabular}{@{}l|l@{}}
\toprule
\multicolumn{1}{c|}{Monday} & \multicolumn{1}{c}{Wednesday} \\
\hline
(3/7) Project introduction     & (3/9) Git workflow I \\
(3/14) Follow-up discussion    & (3/16) Git workflow II \\
\emph{\hspace{12mm} Spring break}  & \emph{\hspace{12mm} Spring break}\\
(3/28) Start final project     & (3/30) TBD\\
(4/4) Project check in         & (4/6) TBD\\
(4/11) Project check in        & (4/13) TBD\\
(4/18) Project check in        & (4/20) TBD\\
(4/25) Project check in        & (4/27) TBD\\
(5/2) Project check in         & (5/4) TBD\\
\emph{\hspace{12mm} RRR week}  & \emph{\hspace{12mm} RRR week}\\
\emph{\hspace{12mm} Final week}  & \emph{\hspace{12mm} Final week}\\
%(5/9) RRR week                 & (5/11) RRR week\\
%(5/16) Final week              & (5/18) Final week\\
\bottomrule
\end{tabular}
\end{table}

On Monday, March 7th during class, Professor Larry Tecott and Dr. Chris Hillar
will provide a general introduction to the project from the perspective of a
behavioral neuroscientist (Professor Tecott) as well as from the perspective of
a computational data scientist (Dr. Hillar).  Part of the class time will be
spent in an open discussion with the students.  Professor Tecott will also lead
a follow-up discussion on Monday, March 14th during class.

\section{Project Deliverables}

There will be two main project deliverables: a Python package and a
final self-evaluation.  The entire class will be responsible for
the Python package and their will be one grade for the final project.
Each student will be required to submit a final self-evaluation
and to schedule a 30 minute meeting with me during finals to discuss
your evaluations.

\subsection{Python package}

I will be responsible for creating the initial project infrastructure
on GitHub.  If there are code or infrastructure issues that the
class can't agree on, I will be responsible for making the final
decision.  However, before I intervene, you will need to carefully
think through the issue and prepare arguments for and against
any decision you wish me to make.

While you will be responsible for the majority of design decisions,
I will require that the Python package have:

\begin{itemize}
\item an automated test suite with a reasonably high test coverage,
\item a comprehensive code review process for all contributed code (using
   GitHub's pull request mechanism and continuous integration using
   TravisCI and Coveralls), and
\item extensive, high quality documentation using Sphinx.
\end{itemize}

I will maintain the official project repository and will be the primary
gatekeeper (i.e., I will be the one primarily responsible for merging all pull
requests).  However, I will require that pull requests undergo a high level of
review and scrutiny before I will consider merging it.  As a class, we will
develop a code review process, which will include (among other things) program
correctness, test coverage, code readability, and style consistency.

\subsection{Self performance evaluation}

\end{document}
