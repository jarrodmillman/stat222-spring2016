\documentclass[11pt]{article}
\usepackage[margin=0.75in]{geometry}            % See geometry.pdf to learn the layout options. There are lots.
\geometry{letterpaper}                                  % ... or a4paper or a5paper or ... 
%\geometry{landscape}                           % Activate for rotated page geometry
%\usepackage[parfill]{parskip}                  % Activate to begin paragraphs with an empty line rather than an indent
\usepackage{graphicx}                           % Use pdf, png, jpg, or eps§ with pdflatex; use eps in DVI mode
                                                                % TeX will automatically convert eps --> pdf in pdflatex                
\usepackage{amssymb}
\usepackage{upquote}

%-----------------------------------------------------------------------------
% Special-purpose color definitions (dark enough to print OK in black and white)
\usepackage{color}
% A few colors to replace the defaults for certain link types
\definecolor{orange}{cmyk}{0,0.4,0.8,0.2}
\definecolor{darkorange}{rgb}{.71,0.21,0.01}
\definecolor{darkgreen}{rgb}{.12,.54,.11}
%-----------------------------------------------------------------------------
% The hyperref package gives us a pdf with properly built
% internal navigation ('pdf bookmarks' for the table of contents,
% internal cross-reference links, web links for URLs, etc.)
\usepackage{hyperref}
\hypersetup{pdftex, % needed for pdflatex
  breaklinks=true, % so long urls are correctly broken across lines
  colorlinks=true,
  urlcolor=blue,
  linkcolor=darkorange,
  citecolor=darkgreen,
}


\title{Wireless Sensor Network Project\\
  Stat 222, Spring 2016}

\author{
  Your Name\\
  \texttt{your github account}
}


\begin{document}
\maketitle

\abstract{Include your abstract here.  Your report should be no more than 12
  pages, including figures.}


\section{Introduction}

Briefly describe \cite{tolle2005macroscope, yang2003redwoods} as well as
provide an overview of your report.

\section{The Data}


\subsection{Data Collection}


\subsection{Data Cleaning}


\subsection{Data Exploration}


\section{Graphical Critique}


\section{Findings}


\subsection{First finding}

Describe it using figure~\ref{fig:find1}

\begin{figure}
  \centering
    put a pretty picture here
    %\includegraphics[width=0.5\textwidth]{path_to_fig}
  \caption{put a caption here}
  \label{fig:find3}
\end{figure}


\subsection{Second finding}

Describe it using figure~\ref{fig:find2}

\begin{figure}
  \centering
    put a pretty picture here
    %\includegraphics[width=0.5\textwidth]{path_to_fig}
  \caption{put a caption here}
  \label{fig:find2}
\end{figure}


\subsection{Third finding}

Describe it using figure~\ref{fig:find3}

\begin{figure}
  \centering
    put a pretty picture here
    %\includegraphics[width=0.5\textwidth]{path_to_fig}
  \caption{put a caption here}
  \label{fig:find3}
\end{figure}


\section{Discussion}


\section{Conclusion}

% the `Acknowledgments` section is optional.
\section*{Acknowledgments}
If you discussed this project with any of your classmates, please thank
them here.

\bibliographystyle{plain}
\bibliography{sensor}

\end{document}
