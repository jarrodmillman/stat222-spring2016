\documentclass[11pt, oneside]{article}   	% use "amsart" instead of "article" for AMSLaTeX format
\usepackage{geometry}                		% See geometry.pdf to learn the layout options. There are lots.
\geometry{letterpaper}                   		% ... or a4paper or a5paper or ... 
%\geometry{landscape}                		% Activate for rotated page geometry
%\usepackage[parfill]{parskip}    		% Activate to begin paragraphs with an empty line rather than an indent
\usepackage{graphicx}				% Use pdf, png, jpg, or eps§ with pdflatex; use eps in DVI mode
								% TeX will automatically convert eps --> pdf in pdflatex		
\usepackage{amssymb}
\usepackage{upquote}

%-----------------------------------------------------------------------------
% Special-purpose color definitions (dark enough to print OK in black and white)
\usepackage{color}
% A few colors to replace the defaults for certain link types
\definecolor{orange}{cmyk}{0,0.4,0.8,0.2}
\definecolor{darkorange}{rgb}{.71,0.21,0.01}
\definecolor{darkgreen}{rgb}{.12,.54,.11}
%-----------------------------------------------------------------------------
% The hyperref package gives us a pdf with properly built
% internal navigation ('pdf bookmarks' for the table of contents,
% internal cross-reference links, web links for URLs, etc.)
\usepackage{hyperref}
\hypersetup{pdftex, % needed for pdflatex
  breaklinks=true, % so long urls are correctly broken across lines
  colorlinks=true,
  urlcolor=blue,
  linkcolor=darkorange,
  citecolor=darkgreen,
}

\usepackage{booktabs}


\title{Stat 222: Sensor Project}
\date{}							% Activate to display a given date or no date

\begin{document}
\maketitle

\section{Data Description}

For this project, your primary data source will be from a wireless sensor
network, which captured spatial and temporal information (e.g., temperature,
humidity) from the microclimate around a coastal redwood tree
\cite{tolle2005macroscope, yang2003redwoods}.  Wireless sensor networks (WSNs)
are becoming increasingly prevalent in a growing number of areas including
health care, environmental and industrial monitoring, as well as home
automation.

WSNs are typically composed of hundreds to thousands of nodes connected by a
wireless network topology (e.g., star or mesh).  Each node is composed of a
radio tranceiver and antenna, a microcontroller (running a minimal, embedded
Operating System such as TinyOS), one or more sensors, and a power source
(e.g., a battery or a way to harvest solar, kinetic, or thermal energy).  A
central challenge (or goal) of WSN design is to create low cost, tiny sensor
nodes.  These size and cost constraints in turn place heavy constraints on
resources such as the battery, memory, computational capacity, and
communication bandwidth.  Often energy is the scarcest resource.  WSNs are
often designed to be resilient to node failure, capable of withstanding harsh
environmental conditions, and easy to install and use. 

A central goal for this project is to give you more hands-on experience working
with the types of messy, incomplete, and inconsistent data that you will
encounter in real world applications.  It is also designed to give you more
time to develop your ability to understand and critique statistical graphics as
well as to provide you an opportunity to practice designing good graphics to
convey information and reveal patterns.  Finally, the project will provide you
with the opportunity to practice reading and writing about applied statistical
data analysis.


\section{Your Assignment}

This is an individual project.  While you may talk to other students in the
class about the assignment, you will be responsible for producing your own
work.  This includes all code, figures, and text.

You are free to use either R or Python.

\subsection*{Exploration of Data}

Your first task will be to check the data quality.  This involves understanding
the data collection method as well as all the potential data entry issues
(e.g., missing values, errors in data). Please read the paper to understand how
the sensor works, and write a paragraph to discuss the measurement of each
variable you find interesting in the data.  Please have at least 3 variables
(related to your findings) in your report.

Bearing the data quality in mind, your second task will be data cleaning. This
data set is quite raw---it contains some gross outliers, inconsistencies, and
lots of missing values. Read the ``Outlier rejection'' section in the paper
carefully and critically.  You will need to do some cleaning of the data but do
\textbf{not} blindly follow their method. Record in your report the steps you
take and any evidence (i.e., summary statistics and EDA plots) you use to
support them.

Next, think of some questions you would like to ask of the data and use R or
Python to answer them graphically. Try to show what interesting findings can be
gained from the data. You may show general patterns or anecdotal events.  Using
the entire dataset may be challenging. Try just a subset of sensor nodes or a
day's worth of data. You may also need to jitter (i.e., adding a small amount
of randomness) to your data, so that you don't overplot or overlap elements.
Again record in your report your process---include plots you make. Don't be
afraid to try methods that are new to you and be critical of your own graphics.

\subsection*{Graphical Critique}

Critique the plots in Figures 3 and 4 of the original paper. You should
make sure to incorporate the material that Deb Nolan presented in class as well
as the assigned readings.  In particular, you should carefully consider and
address the following questions:

\begin{enumerate}
\item What is the data?  What observations and what variables are included
  (or not)?
\item What is the message? What questions does the graphic try to answer?
  Does the graphic answer them successfully? Does it raise any questions not
  addressed in the text?
\item How would you improve it?  Be specific. Discuss both minor tweaks that
  would improve on the existing graphic as well as alternative graphics
  including possibly additional data, which may be better suited for the
  questions at hand.
\end{enumerate}

\subsection*{Presenting findings}

Choose three interesting findings from your exploratory data analysis and
produce a publication quality graphic for each along with a short caption of
what each shows. I expect to see very polished graphics. Think carefully about
your use of color, labeling, shading, transparency, etc. Again, you should be
sure to review the material from the readings and the guest lecture.

\subsection*{Timeline and logistics}

Here is the tentative schedule:

\begin{table}[h]
\centering
\begin{tabular}{@{}l|l@{}}
\toprule
\multicolumn{1}{c|}{Monday} & \multicolumn{1}{c}{Wednesday} \\
\hline
(2/8) Start Sensor Project    & (2/10) Social networks \\
\emph{\hspace{12mm} No class} & (2/17) TBD \\
(2/22) Twitter posters        & (2/24) Financial data \\
(2/29) Sensor report          & \\
\bottomrule
\end{tabular}
\end{table}

We will \textbf{not} discuss this project in class.  This project will require
a significant effort on your part.  You will need to understand the paper as
well as the data.  Identifying and dealing with data quality issues will
involve carefully investigation and thought.  Since you will be simulatenously
working on your Twitter project, you will need to be careful with your time and
will need to communicate clearly with your Twitter project team about deadlines
and work schedules.

\section{Report Details}

I've created a Git repository for you with the following structure:

\begin{verbatim}
sensor
|-- data
|   |-- mote-location-data.txt
|   |-- README.md
|   |-- sonoma-data-all.csv
|   |-- sonoma-data-log.csv
|   |-- sonoma-data-net.csv
|   `-- variable_key.txt
`-- report
    |-- Makefile
    |-- sensor.bib
    `-- sensor.tex

2 directories, 9 files
\end{verbatim}

The files of
interest are \texttt{sonoma-data-all.csv} and \texttt{mote-location-data.txt}.

I have provided a template in your individual class repositories for the
writeup. Since the template is intended, in part, to make grading easier,
please do not deviate from it without good reason.  Please restrict your
writeup to twelve pages, including figures. This is a strict limit.

\bibliographystyle{plain}
\bibliography{sensor}

\end{document}  
