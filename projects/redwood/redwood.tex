\documentclass[11pt, oneside]{article}   	% use "amsart" instead of "article" for AMSLaTeX format
\usepackage{geometry}                		% See geometry.pdf to learn the layout options. There are lots.
\geometry{letterpaper}                   		% ... or a4paper or a5paper or ... 
%\geometry{landscape}                		% Activate for rotated page geometry
%\usepackage[parfill]{parskip}    		% Activate to begin paragraphs with an empty line rather than an indent
\usepackage{graphicx}				% Use pdf, png, jpg, or eps§ with pdflatex; use eps in DVI mode
								% TeX will automatically convert eps --> pdf in pdflatex		
\usepackage{amssymb}

%-----------------------------------------------------------------------------
% Special-purpose color definitions (dark enough to print OK in black and white)
\usepackage{color}
% A few colors to replace the defaults for certain link types
\definecolor{orange}{cmyk}{0,0.4,0.8,0.2}
\definecolor{darkorange}{rgb}{.71,0.21,0.01}
\definecolor{darkgreen}{rgb}{.12,.54,.11}
%-----------------------------------------------------------------------------
% The hyperref package gives us a pdf with properly built
% internal navigation ('pdf bookmarks' for the table of contents,
% internal cross-reference links, web links for URLs, etc.)
\usepackage{hyperref}
\hypersetup{pdftex, % needed for pdflatex
  breaklinks=true, % so long urls are correctly broken across lines
  colorlinks=true,
  urlcolor=blue,
  linkcolor=darkorange,
  citecolor=darkgreen,
}

\usepackage{booktabs}


\title{Stat 222: Redwood Project}
\date{}							% Activate to display a given date or no date

\begin{document}
\maketitle

\section{Your Assignment}

This is an individual project.  While you may talk to other students in the
class about the assignment, you will be responsible for producing your own
work.  This includes all code, figures, and text.

The data for this lab is taken from Tolle et al., which can be found
\texttt{here}. You should read this paper before doing the lab and understand
the source of the data.

I have provided a template in your individual class repositories for the
writeup. Since the template is intended, in part, to make grading easier,
please do not deviate from it without good reason.  Please restrict your
writeup to twelve pages, including figures. This is a strict limit.

\section{Exploration of Data}

The original data can be found in the lab1 folder in bCourses. The files of
interest are sonoma-data-all.csv and mote-location-data.txt. The goal of this
task is to simulate receiving data in a collaboration. Your first goal is to
explore the data on your own. Try to understand how variables behave, and what
their relationships are. This also involves carefully cleaning the data set. Do
not take data consistency or correctness for granted. The following is a
suggestion on how you might proceed.

Your first task will be to check the data quality and explicitly address the
issues we discussed in class, such as the data collection method and data entry
issues (e.g. missing values, errors in data, etc). Please read the paper to
understand how the sensor works, and write a paragraph to discuss the
measurement of each variable you find interesting in the data. Please have at
least 3 variables in your report, and those variables should be related to your
findings in 1.3.

Bearing the data quality in mind, your second task will be data cleaning. This
data set is quite raw - it contains some gross outliers, inconsistencies, and
lots of missing values. Read the ``Outlier rejection'' section in the paper
carefully and critically.  You will need to do some cleaning of the data but
don’t blindly follow their method. Record in your report the steps you take and
any evidence you use to support them.

Next, think of some questions you would like to ask of the data and use R to
answer them graphically. Try to show what interesting findings can be gained
from the data. You may show general patterns or anecdotal events. Experiment
with linked plots. Using the entire dataset may be challenging. Try just a
subset of sensor nodes or a day's worth of data. Again record in your report
your process---include plots you make. Don't be afraid to try methods that are
new to you and be critical of your own graphics.

\section{Graphical Critique}

Critique the plots in Figures 3 \& 4. What questions did they try to answer?
Did they answer them successfully? Did they raise any questions not addressed
in the text? Would you change them at all?


\section{Presenting findings}

Choose three of your interesting findings and produce a publication quality
graphic for each along with a short caption of what each shows. This is where I
expect to see very polished graphics. Think carefully about use of color,
labeling, shading, transparency, etc. This is your chance to do something
innovative. If you are feeling bored or ambitious consider doing something
dynamic or interactive.


\section{Discussion}

Did the data size restrict you in any way? Discuss a new aspect of large data
sets from the lab.

\end{document}  
