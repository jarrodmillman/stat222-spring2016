\documentclass[11pt, oneside]{article}   	% use "amsart" instead of "article" for AMSLaTeX format
\usepackage{geometry}                		% See geometry.pdf to learn the layout options. There are lots.
\geometry{letterpaper}                   		% ... or a4paper or a5paper or ... 
%\geometry{landscape}                		% Activate for rotated page geometry
%\usepackage[parfill]{parskip}    		% Activate to begin paragraphs with an empty line rather than an indent
\usepackage{graphicx}				% Use pdf, png, jpg, or eps§ with pdflatex; use eps in DVI mode
								% TeX will automatically convert eps --> pdf in pdflatex		
\usepackage{amssymb}
\usepackage{upquote}

%-----------------------------------------------------------------------------
% Special-purpose color definitions (dark enough to print OK in black and white)
\usepackage{color}
% A few colors to replace the defaults for certain link types
\definecolor{orange}{cmyk}{0,0.4,0.8,0.2}
\definecolor{darkorange}{rgb}{.71,0.21,0.01}
\definecolor{darkgreen}{rgb}{.12,.54,.11}
%-----------------------------------------------------------------------------
% The hyperref package gives us a pdf with properly built
% internal navigation ('pdf bookmarks' for the table of contents,
% internal cross-reference links, web links for URLs, etc.)
\usepackage{hyperref}
\hypersetup{pdftex, % needed for pdflatex
  breaklinks=true, % so long urls are correctly broken across lines
  colorlinks=true,
  urlcolor=blue,
  linkcolor=darkorange,
  citecolor=darkgreen,
}

\usepackage{booktabs}


\title{Stat 222: Redwood Project}
\date{}							% Activate to display a given date or no date

\begin{document}
\maketitle

\section{Data Description}

For this project, your primary data source will be from a wireless sensor
network, which captured spatial and temporal information (e.g., temperature,
humidity) from the microclimate around a coastal redwood tree
\cite{tolle2005macroscope}.  Wireless sensor networks are becoming increasingly
prevalent in a growing number of areas including health care, environmental and
industrial monitoring, as well as home monitoring and automation.

An important goal for this project is to give you more hands-on experience
working with the types of messy, incomplete, and inconsistent data that you
will encounter in real world applications.  It is also designed to give you
more time to develop your ability to understand and critique statistical
graphics as well as to provide you an opportunity to practice designing good
graphics to convey information and reveal patterns.  Finally, the project will
provide you with the opportunity to practice reading and writing about applied
statistical data analysis.


\section{Your Assignment}

This is an individual project.  While you may talk to other students in the
class about the assignment, you will be responsible for producing your own
work.  This includes all code, figures, and text.

I've created a Git repository for you with the following structure:

\begin{verbatim}
redwood
|-- data
|   |-- mote-location-data.txt
|   |-- README.md
|   |-- sonoma-data-all.csv
|   |-- sonoma-data-log.csv
|   |-- sonoma-data-net.csv
|   `-- sonoma-dates.Rda
`-- redwoods-sensys05.pdf

1 directory, 7 files
\end{verbatim}

The files of
interest are \texttt{sonoma-data-all.csv} and \texttt{mote-location-data.txt}.

\subsection*{Exploration of Data}

Your first task will be to check the data quality and explicitly address the
issues we discussed in class, such as the data collection method and data entry
issues (e.g. missing values, errors in data, etc). Please read the paper to
understand how the sensor works, and write a paragraph to discuss the
measurement of each variable you find interesting in the data. Please have at
least 3 variables in your report, and those variables should be related to your
findings.

Bearing the data quality in mind, your second task will be data cleaning. This
data set is quite raw---it contains some gross outliers, inconsistencies, and
lots of missing values. Read the ``Outlier rejection'' section in the paper
carefully and critically.  You will need to do some cleaning of the data but do
\textbf{not} blindly follow their method. Record in your report the steps you
take and any evidence you use to support them.

Next, think of some questions you would like to ask of the data and use R or
Python to answer them graphically. Try to show what interesting findings can be
gained from the data. You may show general patterns or anecdotal events.
Experiment with linked plots. Using the entire dataset may be challenging. Try
just a subset of sensor nodes or a day's worth of data. Again record in your
report your process---include plots you make. Don't be afraid to try methods
that are new to you and be critical of your own graphics.

\subsection*{Graphical Critique}

Critique the plots in Figures 3 \& 4. What questions did they try to answer?
Did they answer them successfully? Did they raise any questions not addressed
in the text? Would you change them at all?

\subsection*{Presenting findings}

Choose three of your interesting findings and produce a publication quality
graphic for each along with a short caption of what each shows. This is where I
expect to see very polished graphics. Think carefully about use of color,
labeling, shading, transparency, etc. This is your chance to do something
innovative. If you are feeling bored or ambitious consider doing something
dynamic or interactive.

\subsection*{Discussion}

Did the data size restrict you in any way? Discuss a new aspect of large data
sets from the lab.

\subsection*{Timeline and logistics}

Here is the tentative schedule:

\begin{table}[h]
\centering
\begin{tabular}{@{}l|l@{}}
\toprule
\multicolumn{1}{c|}{Monday} & \multicolumn{1}{c}{Wednesday} \\
\hline
(2/8) Start Redwood project   & (2/10) Workflow I \\
\emph{\hspace{12mm} No class} & (2/17) Workflow II \\
(2/22) Poster presentations   & (2/24) Workflow III \\
(2/29) Redwood report         & \\
\bottomrule
\end{tabular}
\end{table}

You should spend

\section{Report Details}

I have provided a template in your individual class repositories for the
writeup. Since the template is intended, in part, to make grading easier,
please do not deviate from it without good reason.  Please restrict your
writeup to twelve pages, including figures. This is a strict limit.

\bibliographystyle{plain}
\bibliography{redwood}

\end{document}  
