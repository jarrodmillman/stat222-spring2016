% Document settings
\documentclass[11pt]{article}
\usepackage[margin=1in]{geometry}
\usepackage[pdftex]{graphicx}
\usepackage{multirow}
\usepackage{setspace}

%-----------------------------------------------------------------------------
% Special-purpose color definitions (dark enough to print OK in black and white)
\usepackage{color}

% A few colors to replace the defaults for certain link types
\definecolor{orange}{cmyk}{0,0.4,0.8,0.2}
\definecolor{darkorange}{rgb}{.71,0.21,0.01}
\definecolor{darkgreen}{rgb}{.12,.54,.11}

%-----------------------------------------------------------------------------
% The hyperref package gives us a pdf with properly built
% internal navigation ('pdf bookmarks' for the table of contents,
% internal cross-reference links, web links for URLs, etc.)
\usepackage{hyperref}

\hypersetup{pdftex,  % needed for pdflatex
  breaklinks=true,  % so long urls are correctly broken across lines
  colorlinks=true,
  urlcolor=blue,
  linkcolor=darkorange,
  citecolor=darkgreen,
  }


\usepackage{url}

\pagestyle{plain}
\setlength\parindent{0pt}

\begin{document}

% Course information
\begin{tabular}{ l l }
  \multirow{3}{*}{\includegraphics[height=1.25in,width=1.25in]{../_fig/ucberkeleyseal_874_540.eps}}
  & \LARGE Statistics 222 --- Spring 2016 Syllabus\\
  & \LARGE Masters of Statistics Capstone Project \\\\
  & \Large CCN: 87789 \\
  & \Large Class meets MW 9--11A in 340 EVANS \\
\end{tabular}
\vspace{10mm}

% Professor information
\begin{tabular}{ l l }
  \multirow{6}{*} & \large Instructor: K. Jarrod Millman \\
  & \large \url{http://www.jarrodmillman.com} \\
  & \large Office Location: 210Q Barker Hall \\
  & \large Office Hours: TBA \\
\end{tabular}
\hspace{20mm}
% GSI information
\begin{tabular}{ l l }
  \multirow{6}{*} & \large GSI: Johnny Hong \\
  & \large \url{http://} \\
  & \large Office Location: TBA \\
  & \large Office Hours: TBA \\
\end{tabular}
\vspace{5mm}
\begin{center} I reserve the right to make changes to the syllabus.\\
\end{center}

% Course details
\textbf {\large \\ Course Description:}
In this course you will develop a portfolio of data analysis projects,
culminating in a collaborative group project in which you'll work with an
academic or industry partner. The course is somewhat unique in that it's
organized around data sets and not around a prespecified set of lecture topics.
However, along the way, we'll cover a variety of methods used in modern applied
statistics, we'll explore computational tools for working with non-standard
data formats and doing reproducible research, and we'll practice the
communication skills you'll need to be a successful practicing statistician.

\textbf {\large \\ Course Structure:}
In the first part of the course, everyone will be working on the same data
analysis projects, some individually and some in groups. For each of the group
projects, you'll be assigned a team of 3-4 students.  In the second part of
the course, you'll again be working in a team, and you'll also have an industry
partner who is providing the data set. For this project, each team will turn in
one final report.

\textbf {\large \\ Prerequisites:}
Statistics 201A-201B, 243. Restricted to students who have been admitted to
the one-year Masters Program in Statistics beginning fall 2012 or later.

\textbf {\large \\ Credit Hours:} 4

\textbf {\large \\ Text(s):} 
There is no required textbook for this course.  I will be pointing you to a lot
of different resources as we go, and you will also be collecting them yourself
and sharing them with the class.

However, I strongly recommend:

\vspace{2mm}

Friedman, J., Hastie, T., \& Tibshirani, R. (2001). \emph{The Elements of Statistical Learning}
(Vol. 1). Springer, Berlin: Springer Series in Statistics.

\vspace{2mm}

You can find online via \href{http://link.springer.com/book/10.1007\%2F978-0-387-21606-5}{SpringerLink}. \\

\textbf {\large \\ Grading:}
\hspace*{40mm}
\begin{tabular}{ l l }
Participation  & 5\% \\
Quizzes  & 5\% \\
Readings & 5\% \\
Individual report  & 10\% \\
Small projects & 20\% \\
Final project  & 55\%
\end{tabular} \\\\

Quizzes will be held during class.  I will drop your lowest score.

For each assigned reading, you will submit a 2 paragraph report by 21:00 on the
Thursday it is due.  Unless otherwise specified, the first paragraph should
summarize the reading.  The second paragraph should briefly explore something
that interested you (e.g., you may wish to focus on one aspect of the paper in
more depth, you may wish to discuss something in the reading that you disagree
with).

There will be 2 small projects and 1 individual project. You will be provided a
rubric at the start of each project.  You may discuss the individual projects
with your classmates, but you will be required to work on it independently.

Your final project report will be graded on a 0-100 scale. I'll give you a
rubric ahead of time.

%\newpage

\textbf{\large \\ Course Policies:} \\

\textbf{Attendance and behavior in class}: You are expected to attend all lectures
and labs.  Any known or potential extracurricular conflicts should be discussed
in person with me during the first two weeks of the semester, or as
soon as they arise. \textbf{Cellphones} are to be turned off during class time.
\textbf{Laptop} use during class will often be required, but should be
used for course work only (i.e., not for surfing the web).\\

\textbf{Submission of assignments}: Assignments will be accepted by electronic
submission to GitHub only.  There will be no makeup quizzes. No
late reading reports or homeworks will be accepted. \\ % Grades of Incomplete will be granted
%only for dire medical or personal emergencies that cause you to miss the final project
%presentation, and only if your work up to that point has been satisfactory.\\

\textbf{Academic integrity}:
Any work submitted by you and that bears your name is presumed to be your own
original work that has not previously been submitted for credit in another
course. While you are permitted to discuss the assignment for the individual project,
both the code and writeup must be your own.  In particular, discussing your
code with another student is acceptable, whereas simply giving him or her your
own code is not.  This is different from the group projects, in which what you
turn in will be jointly produced by the entire team, and sharing of code is
encouraged. If you are not clear about the expectations for completing any
particular assignment, be sure to seek clarification from the instructor.  Any
evidence of cheating or plagiarism will be subject to disciplinary action.
Please read the Honor Code (\url{http://asuc.org/honorcode/index.php})
carefully.\\


\textbf{Students with disabilities}: If you need accommodations, please make
arrangements in a timely manner through DSP.\\

%\noindent\textbf{Important Dates}:
%\begin{center} \begin{minipage}{5in}
%\begin{flushleft}
%Form teams \dotfill Sept. 22\\
%Homework 1 \dotfill Sept. 24\\
%Project proposal \dotfill Oct. 1\\
%Homework 2 \dotfill Oct. 26\\
%Progress presentation \dotfill Nov. 12\\
%Draft report \dotfill Nov. 12\\
%BIC tour \dotfill Nov. 23\\
%Project presentation \dotfill Dec. 1 \& 3\\
%Final report \dotfill Dec. 14\\
%\end{flushleft}
%\end{minipage}
%\end{center}

\end{document}
